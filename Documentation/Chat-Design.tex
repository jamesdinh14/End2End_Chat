\documentclass{article}
\usepackage[utf8]{inputenc}
\begin{document}
End-to-End Secure Chat\newline

\textbf{Team Insecurity}: James Dinh and Hsi You Huang\newline

\textbf{Application Properties}\newline

\begin{enumerate}
	\item \textbf{Features}: The chat is a one-on-one system, with a middleman. The two
     “participants” are the clients. The middleman is the server. One client sends a message to the
     server; the server receives that message and passes it on the other client; the other client 
     receives the message. The clients do not have to be active at the same time in order to receive
     messages.
	\item \textbf{Language}: Java
    \item \textbf{Platform}: Web app, Amazon AWS, Spring framework for Java, AES-256 encryption.
    \item \textbf{RESTful server}:
    	\begin{itemize}
    		\item The server will manage each User of the chat. Each user will have its own ID and ?
         	\item POST will create new User with a new ID and ?.
            \item GET will read the information of user based on ID.
            \item PUT will update info on each user.
            \item DELETE will delete the user from the server.
            \item Above functions work with HTTP.
        \end{itemize}
	\item \textbf{Assets}
    	\begin{itemize}
        	\item Client
            \item Server
            \item User information
			\item Message
        \end{itemize} 
    \item \textbf{Stakeholder}
    	\begin{itemize}
        	\item Users of the chat system
        \end{itemize}
    \item \textbf{Adversarial Models}:
    	\begin {itemize}
  
		    \item Outsider: Outsider can be passive or active. Active attacker tries to temper
             network traffic to intercept communication. Passive attacker eavesdrop on the 
             communication.
				
		    \item Insider: Insider have access to our communication network, and can pretend to be
             one of the client to the server, or temper with the communication channel to their 
             advantages.				
        \end {itemize}
    \item \textbf {Possible Vulnerabilities}:
    	\begin {itemize}
        	\item Man-in-the-middle attack
			\item Brute force attack
			\item Work station hijack
			\item Encryption cryptanalysis
			\item Keylogger
			\item 
        \end {itemize}
    \item \textbf {Related Previous Work}:
        \begin {itemize}
            \item WhatsApp
            \item Signal
        \end {itemize}
    \item \textbf {Solution:} The clients will hold a pair of keys that will encrypt and decrypt
     messages on their end. 
    Example: Sender client sends a message to its receiver client. Sender will type up their message
    and “send” it. Before the message leaves the sender, the message is encrypted and then it gets
    sent to the server. The server will receive the encrypted message and then passes the message
    along to the receiver. The receiver will receive the encrypted message, decrypt it, and then the
     client “receives” the message (they are able to read it).\newline
    Against the man-in-the-middle attack, TLS has certificates available for use. We can use the
     certificates to authenticate the two clients. Using this authentication, we can prevent 
     man-in-the-middle attacks.\newline
	Against brute force attack, we will use AES 256 bits encryption to encrypt the messages. To
     brute force an AES-256, it would take far too much money and energy, and far too long to be
      worth while. For this reason, AES-256 would achieve computational security for this project. 
      Using AES-256 also provides solution to the cryptanalysis vulnerability, as there are no known
       cryptanalsys	against it.\newline
	
	\item \textbf{Analysis}:
	\begin {itemize}
		\item Confidentiality\newline
            The AES encryption will offer strong protection against an outside adversary attempting
            to gain access to the client's messages. 
		\item Integrity\newline
            Against an insider man-in-the-middle adversary, certificates will be used to verify the 
            identify of the message source. If a man-in-the-middle (MITM) adversary is detected,
            they will be unable to receive the message and any messages sent by the MITM will not be
            received by the clients.
		\item Authentication\newline
            The certificates will provide a form of authentication. Messages received by a client 
            will have its certificate checked. If it passes, the message is confirmed to have been 
            sent by the other client. If the certificate doesn't pass, the message is not confirmed
            to be from the other client.
    \end {itemize}
	
\end{enumerate}

\end{document}
