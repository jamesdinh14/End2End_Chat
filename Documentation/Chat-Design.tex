\documentclass{article}
\usepackage[utf8]{inputenc}
\begin{document}
End-to-End Secure Chat\newline

\textbf{Team Insecurity}: James Dinh and Hsi You Huang\newline

\textbf{Application Properties}\newline

\begin{enumerate}
	\item \textbf{Features}: The chat is a one-on-one system, with a middleman. The two
     “participants” are the clients. The middleman is the server. One client sends a message to the
     server; the server receives that message and passes it on the other client; the other client 
     receives the message. The clients do not have to be active at the same time in order to receive
     messages.
	\item \textbf{Language}: Java
    \item \textbf{Platform}: Web app, Amazon AWS, Spring framework for Java
    \item \textbf{RESTful server}:
    	\begin{itemize}
    		\item The server will manage each User of the chat. Each user will have its own ID and ?
         	\item POST will create new User with a new ID and ?.
            \item GET will read the information of user based on ID.
            \item PUT will update info on each user.
            \item DELETE will delete the user from the server.
            \item Above functions work with HTTP.
        \end{itemize}
	\item \textbf{Assets}
    	\begin{itemize}
        	\item Client
            \item Server
            \item User information
        \end{itemize} 
    \item \textbf{Stakeholder}
    	\begin{itemize}
        	\item Users of the chat system
        \end{itemize}
    \item \textbf{Adversarial Models}:
    	\begin {itemize}
            \item Eavesdropper
		    \item Outsider
		    \item Insider
        \end {itemize}
    \item \textbf {Possible Vulnerabilities}:
    	\begin {itemize}
        	\item Man-in-the-middle attack
        \end {itemize}
    \item \textbf {Related Previous Work}:
    \item \textbf {Solution:} The clients will hold a “key” that will encrypt and decrypt messages
     on their end. 
    Example: Sender client sends a message to its receiver client. Sender will type up their message
    and “send” it. Before the message leaves the sender, the message is encrypted and then it gets
    sent to the server. The server will receive the encrypted message and then passes the message
    along to the receiver. The receiver will receive the encrypted message, decrypt it, and then the
     client “receives” the message (they are able to read it).\newline
    Against the man-in-the-middle attack, TSL has certificates available for use. We can use the
     certificates to authenticate the two clients. Using this authentication, we can prevent 
     man-in-the-middle attacks.
	\item \textbf{Analysis}:
\end{enumerate}

\end{document}
